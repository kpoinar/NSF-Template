\section*{\hfil Data Management Plan \hfil}
\vspace{-16pt}
\noindent\hrulefill

%Proposals must include a supplementary document of no more than two pages labeled ``Data Management Plan". This supplementary document should describe how the proposal will conform to NSF policy on the dissemination and sharing of research results (see AAG Chapter VI.D.4)
%Standard templated based on NSF PAPPG 24-1 guidelines, be sure to review the solicitation and any special guidance from the directorate, office, division, or program.

\noindent{\large\textbf{Expected Data}}
 
\noindent Describe the types of data, samples, physical collections, software, or other material to be produced in the course of the project. 
 
 
 
\bigskip
\noindent{\large\textbf{Data Format}}
 
\noindent Describe the format in which the data or products are stored (e.g., hardcopy notebook and/or instrument outputs, ASCII, html, jpeg or other formats). Where data are stored in unusual or not generally accessible formats, explain how the data may be converted to a more accessible format or otherwise made available to interested parties. You may also comment on the current or anticipated need for interested parties outside of your laboratory to access your primary data. 
 
 
 
\bigskip
\noindent{\large\textbf{Access to Data and Data Sharing Practices and Policies}}
 
\noindent Access to data refers to data made accessible without explicit request from the interested party, for example, those posted on a website or made available to a public database. Describe your plans, if any, for providing such general access to data, including websites maintained by your research group, and direct contributions to public databases (e.g., IRIS for seismological data, Cambridge Crystallographic Data Centre, Inorganic Crystal Structure Database, the Protein Data Bank, Space Physics Data Center (SPDF), the National Space Science Data Center (NSSDC), Planetary Data System, etc.). Also note if you submit your data in the form of tables, graphs, computer code or other format to the supplementary materials sections of peer-reviewed journals. Describe your practice or policies regarding the release of data for access, for example, whether data are posted before or after formal publication. 
 
Data sharing refers to the release of data in response to a specific request from an interested party. 
Describe your policies for data sharing, including (if applicable) provisions for protection of intellectual property, national security, or other rights or requirements. 
 
\bigskip
\noindent{\large\textbf{Policies for Re-Use and Re-Distribution}}
 
\noindent Describe your policies regarding the use of data provided via general access or sharing. For example, if you plan to provide data and images on your website, will the website contain disclaimers, or conditions regarding the use of the data in other publications or products? Describe these disclaimers and/or terms of use. 
 
 
\bigskip
\noindent{\large\textbf{Archiving of Data}}
 
\noindent Describe how data will be archived and how preservation of access will be handled. For example, will hardcopy notebooks, instrument outputs, and physical samples be stored in a location where there are safeguards against fire or water damage? Is there a plan to transfer digitized information to new storage media or devices as technological standards or practices change? Will there be an easily accessible index that documents where all archived data are stored and how they can be accessed? How long will data be retained?
